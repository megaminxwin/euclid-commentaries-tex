\chapter[Greek commentators other than Proclus]{Greek commentators on the \emph{Elements} other than Proclus}

\textsc{That} there was no lack of commentaries on the \emph{Elements} before the time of Proclus is evident from the terms in which Proclus refers to them; and he leaves us in equally little doubt as to the value which, in his opinion, the generality of them possessed. Thus he says in one place (at the end of his second prologue)\footnote{}:

``Before making a beginning with the investigation of details, I warn those who may read me not to expect from me the things which have been dinned into our ears \emph{ad nauseam} (διατεθρύληται) by those who have preceded me, viz.\ lemmas, cases, and so forth. For I am surfeited with these things and shall give little attention to them. But I shall direct my remarks principally to the points which require deeper study and contribute to the sum of philosophy, therein emulating the Pythagoreans who even had this common phrase for what I mean `a figure and a platform, but not a figure and sixpence\footnote{}.'\thinspace''

In another place\footnote{} he says: ``Let us now turn to the elucidation of the things proved by the writer of the Elements, selecting the more subtle of the comments made on them by the ancient writers, while cutting down their interminable diffuseness, giving the things which are more systematic and follow scientific methods, attaching more importance to the working-out of the real subject-matter than to the variety of cases and lemmas to which we see recent writers devoting themselves for the most part.''

At the end of his commentary on Eucl.~1.\ Proclus remarks\footnote{} that the commentaries then in vogue were full of all sorts of confusion, and contained no account of \emph{causes}, no dialectical discrimination, and no philosophic thought.

These passages and two others in which Proclus refers to ``the commentators\footnote{}'' suggest that these commentators were numerous. He does not however give many names; and no doubt the only important commentaries were those of Heron, Porphyry, and Pappus.

\textbf{I. Heron.}

Proclus alludes to Heron twice as Heron \emph{mechanicus}\footnote{}, in another place\footnote{} he associates him with Ctesibius, and in the three other passages\footnote{} where Heron is mentioned there is no reason to doubt that the same person is meant, namely Heron of Alexandria. The date of Heron is still a vexed question. In the early stages of the controversy much was made of the supposed relation of Heron to Ctesibius. The best \textsc{ms.}\ of Heron's \emph{Belopoeica} has the heading ``Ηρωνος Κτησιβίου βελοποιϊκά'', and an anonymous Byzantine writer of the tenth century, evidently basing himself on this title, speaks of Ctesibius as Heron's καθηγητής, ``master'' or ``teacher.'' We know of two men of the name of Ctesibius. One was a barber who lived in the time of Ptolemy Euergetes II, i.e.\ Ptolemy VII, called Physcon (died 117~\textsc{b.c.}), and who is said to have made an improved water-organ\footnote{}. The other was a mechanician mentioned by Athenaeus as having made an elegant drinking-horn in the time of Ptolemy Philadelphus (285--247~\textsc{b.c.})\footnote{}. Martin\footnote{} took the Ctesibius in question to be the former and accordingly placed Heron at the beginning of the first century \textsc{b.c.}, say 126--50~\textsc{b.c.} But Philo of Byzantium\footnote{}, who repeatedly mentions Ctesibius by name, says that the first mechanicians had the advantage of being under kings who loved fame and supported the arts. Hence our Ctesibius is more likely to have been the earlier Ctesibius who was contemporary with Ptolemy II Philadelphus.

But, whatever be the date of Ctesibius, we cannot safely conclude that Heron was his immediate pupil. The title ``Heron's (edition of) Ctesibius's Belopoeica'' does not, in fact, justify any inference as to the interval of time between the two works.

We now have better evidence for a \emph{terminus post quem}. The \emph{Metrica} of Heron, besides quoting Archimedes and Apollonius, twice refers to ``the books about straight lines (chords) in a circle'' (ἐν τοῖς περὶ τῶν ἐν κύκλῳ εὐθειῶν). Now we know of no work giving a Table of Chords earlier than that of Hipparchus. We get, therefore, at once, 150~\textsc{b.c.}\ of thereabouts as the \emph{terminus post quem}. But, again, Heron's \emph{Mechanica} quotes a definition of ``centre of gravity'' as given by ``Posidonius, a Stoic'': and, even if this Posidonius lived before Archimedes, as the context seems to imply, it is certain that another work of Heron's, the \emph{Definitions}, owes something to Posidonius of Apamea or Rhodes, Cicero's teacher (135--51~\textsc{b.c.}). This brings Heron's date down to the end of the first century~\textsc{b.c.}, at least.

We have next to consider the relation, if any, between Heron and Vitruvius. In his \emph{De Architectura}, brought out apparently in 14~\textsc{b.c.}, Vitruvius quotes twelve authorities on \emph{machinationes} including Archytas (second), Archimedes (third), Ctesibius (fourth) and Philo of Byzantium (sixth), but does not mention Heron; the differences between them seem on the whole more numerous and important than the resemblances (e.g.\ Vitruvius uses 3 as the value of \(\pi\), while Heron always uses the Archimedean value \(3\frac{1}{7}\)). The inference is that Heron can hardly have written earlier than the first century~\ad

The most recent theory of Heron's date makes him later than Claudius Ptolemy the astronomer (100--178~\ad). The arguments are mainly these. (1) Ptolemy claims as a discovery of his own a method of measuring the distance between two places (as an arc of a great circle on the earth's surface) in the case where the places are neither on the same meridian nor on the same parallel circle. Heron, in his \emph{Dioptra}, speaks of this method as of a thing generally known to experts. (2) The dioptra described in Heron's work is a fine and accurate instrument, much better than anything Ptolemy had at his disposal. (3) Ptolemy, in his work Περὶ ῥοπῶν, asserted that water with water round it has no weight and that the diver, however deep he dives, does not feel the weight of the water above him. Heron, strangely enough, accepts as true what Ptolemy says of the diver, but is dissatisfied with the explanation given by ``some,'' namely that it is because water is uniformly heavy---this seems to be equivalent to Ptolemy's dictum that water in water has no weight---and he essays a different explanation based on Archimedes. (4) It is suggested that the Dionysius to whom Heron dedicated his \emph{Definitions} is a certain Dionysius who was \emph{praefectus urbi} in 301~\ad

On the other hand Heron was earlier than Pappus, who was writing under Diocletian (284--305~\ad), for Pappus alludes to and draws upon the works of Heron. The net result, then, of the most recent research is to place Heron in the third century~\ad\ and perhaps little earlier than Pappus. Heiberg\footnote{} accepts this conclusion, which may therefore, perhaps, be said to hold the field for the present\footnote{}.

That Heron wrote a systematic commentary on the \emph{Elements} might be inferred from Proclus, but it is rendered quite certain by references to the commentary in Arabian writers, and particularly in an-Nairīzī's commentary on the first ten Books of the \emph{Elements}. The \emph{Fihrist} says, under Euclid, that ``Heron wrote a commentary on this book [the \emph{Elements}], endeavouring to solve its difficulties\footnote{}''; and under Heron, ``He wrote: the book of explanation of the obscurities in Euclid\footnote{}\dots.'' An-Nairīzī's commentary quotes Heron by name very frequently, and often in such a way as to leave no doubt that the author had Heron's work actually before him. Thus the extracts are given in the first person, introduced by ``Heron says'' (``Dixit Yrinus'' or ``Heron''); and in other places we are told that Heron ``says nothing,'' or ``is not found to have said anything,'' on such and such a proposition. The commentary of an-Nairīzī is in part edited by Besthorn and Heiberg from a Leiden \textsc{ms.}\ of the translation of the \emph{Elements} by al-Ḥajjāj with the commentary attached\footnote{}. But this \textsc{ms.}\ only contains six Books, and several pages in the first Book, which contain the comments of Simplicius on the first twenty-two definitions of the first Book, are missing. Fortunately the commentary of an-Nairīzī has been discovered in a more complete form, in a Latin translation by Gherardus Cremonensis of the twelfth century, which contains the missing comments by Simplicius and an-Nairīzī's comments on the first ten Books. This valuable work has recently been edited by Curtze\footnote{}.

Thus from the three sources, Proclus, and the two versions of an-Nairīzī, which supplement one another, we are able to form a very good idea of the character of Heron's commentary. In some cases observations given by Proclus without the name of their author are seen from an-Nairīzī to be Heron's; in a few cases notes attributed by Proclus to Heron are found in an-Nairīzī without Heron's name; and, curiously enough, one alternative proof (of \textsc{i}.~25) given as Heron's by Proclus is introduced by the Arab with the remark that he has not been able to discover who is the author.

Speaking generally, the comments of Heron do not seem to have contained much that can be called important. We find

\begin{enumerate}[label=(\arabic*)]
	\item A few general notes, e.g.\ that Heron would not admit more than three axioms.
	\item Distinctions of a number of particular \emph{cases} of Euclid's propositions according as the figure is drawn in one way or in another.
\end{enumerate}

Of this class are the different cases of \textsc{i}.~35, 36, \textsc{iii}.~7, 8 (where the chords to be compared are drawn on \emph{different} sides of the diameter instead of on the same side), \textsc{iii}.~12 (which is not Euclid's, but Heron's own, adding the case of external contact to that of internal contact in \textsc{iii}.~11), \textsc{vi}.~19 (where the triangle in which an additional line is drawn is taken to be the \emph{smaller} of the two), \textsc{vii}.~19 (where he gives the particular case of \emph{three} numbers in continued proportion, instead of four proportionals).

\begin{enumerate}[label=(\arabic*)]
	\setcounter{enumi}{2}
	\item Alternative proofs. Of these there should be mentioned (\emph{a}) the proofs of \textsc{ii}.~1--10 ``without a figure,'' being simply the algebraic forms of proof, easy but uninstructive, which are so popular nowadays, the proof of \textsc{iii}.~25 (placed after \textsc{iii}.~30 and starting from the \emph{arc} instead of the chord), \textsc{iii}.~10 (proved by \textsc{iii}.~9), \textsc{iii}.~13 (a proof preceded by a lemma to the effect that a straight line cannot meet a circle in more than two points). Another class of alternative proof is (\emph{b}) that which is intended to meet a particular \emph{objection} (ἔνστασις) which had been or might be raised to Euclid's construction. Thus in certain cases he avoids \emph{producing} a particular straight line, where Euclid produces it, in order to meet the objection of any one who should deny our right to assume that there is \emph{any space available}\footnote{}. Of this class are Heron's proofs of \textsc{i}.~11, \textsc{i}.~20, and his note on \textsc{i}.~16. Similarly on \textsc{i}.~48 he supposes the right-angled triangle which is constructed to be constructed on the \emph{same} side of the common side as the given triangle is. A third class (\emph{c}) is that which avoids \emph{reductio ad absurdum}. Thus, instead of indirect proofs, Heron gives direct proofs of \textsc{i}.~19 (for which he requires, and gives, a preliminary lemma), and of \textsc{i}.~25.
	\item Heron supplies certain \emph{converses} of Euclid's propositions, e.g.\ converses of \textsc{ii}.~12, 13, \textsc{viii}.~27.
	\item A few additions to, and extensions of, Euclid's propositions are also found. Some are unimportant, e.g.\ the construction of isosceles and scalene triangles in a note on \textsc{i}.~1 the construction of \emph{two} tangents in \textsc{iii}.~17, the remark that \textsc{vii}.~3 about finding the greatest common measure of three numbers can be applied to as many numbers as we please (as Euclid tacitly assumes in \textsc{vii}.~31). The most important extension is that of \textsc{iii}.~20 to the case where the angle at the circumference is greater than a right angle, and the direct deduction from this extension of the result of \textsc{iii}.~22. Interesting also are the notes on \textsc{i}.~37 (on \textsc{i}.~24 in Proclus), where Heron proves that two triangles with two sides of one equal to two sides of the other and with the included angles supplementary are equal, and compares the areas where the sum of the two included angles (one being supposed greater than the other) is less or greater than two right angles, and on \textsc{i}.~47, where there is a proof (depending on preliminary lemmas) of the fact that, in the figure of the proposition, the straight lines \(AL\), \(BK\), \(CF\) meet in a point. After \textsc{iv}.~16 there is a proof that, in a regular polygon with an even number of sides, the bisector of one angle also bisects its opposite and an enunciation of the corresponding proposition for a regular polygon with an odd number of sides.
\end{enumerate}

Van Pesch\footnote{} gives reason for attributing to Heron certain other notes found in Proclus, viz.\ that they are designed to meet the same sort of points as Heron had in view in other notes undoubtedly written by him. These are (\emph{a}) alternative proofs of \textsc{i}.~5, \textsc{i}.~17, and \textsc{i}.~32, which avoid the \emph{producing} of certain straight lines, (\emph{b}) an alternative proof of \textsc{i}.~9 avoiding the construction of the equilateral triangle on the side of \(BC\) opposite to \(A\); (\textsc{c}) partial converses of \textsc{i}.~35--38, starting from the equality of the areas and the fact of the parallelograms or triangles being in the same parallels, and proving that the bases are the same or equal, may also be Heron's. Van Pesch further supposes that it was in Heron's commentary that the proof by Menelaus of \textsc{i}.~25 and the proof by Philo of \textsc{i}.~8 were given.

The last reference to Heron made by an-Nairīzī occurs in the note on \textsc{viii}.~27, so that the commentary of the former must at least have reached that point.

\textbf{II. Porphyry.}

The Porphyry here mentioned is of course the Neo-Platonist who lived about 232--304~\ad\@ Whether he really wrote a systematic commentary on the \emph{Elements} is uncertain. The passages in Proclus which seem to make this probable are two in which he mentions him (1) as having demonstrated the necessity of the words ``not on the same side'' in the enunciation of \textsc{i.}~14\footnote{}, and (2) as having pointed out the necessity of understanding correctly the enunciation of \textsc{i}.~26, since, if the particular injunctions as to the sides of the triangles to be taken as equal are not regarded, the student may easily fall into error\footnote{}. These passages, showing that Porphyry carefully analysed Euclids \emph{enunciations} in these cases, certainly suggest that his remarks were part of a systematic commentary. Further, the list of mathematicians in the \emph{Fihrist} gives Porphyry as having written ``a book on the Elements.'' It is true that Wenrich takes this book to have been a work by Porphyry mentioned by Suidas and Proclus (\emph{Theolog.\ Platon.}), περὶ ἀρχῶν libri \textsc{ii}.\footnote{}

There is nothing of importance in the notes attributed to Porphyry by Proclus.

\begin{enumerate}[label=(\arabic*)]
	\item Three alternative proofs of \textsc{i}.~20, which avoid \emph{producing} a side of the triangle, are assigned to Heron and Porphyry without saying which belonged to which. If the first of the three Heron's, I agree with van Pesch that it is more probable that the two others were both Porphyry's than that the second was Heron's and only the third Porphyry's. For they are similar in character, and the third uses a result obtained in the second\footnote{}.
	\item Porphyry gave an alternative proof of \textsc{i}.~18 to meet a childish objection which is supposed to require the part of \(AC\) equal to \(AB\) to be cut off from \(CA\) and not from \(AC\).
\end{enumerate}

Proclus gives a precisely similar alternative proof of \textsc{i}.~6 to meet a similar supposed objection; and it may well be that, though Proclus mentions no name, this proof was also Porphyry's, as van Pesch suggests\footnote{}.

Two other references to Porphyry found in Proclus cannot have anything to do with commentaries on the \emph{Elements}. In the first a work called the Συμμικτά is quoted, while in the second a philosophical question is raised.

\textbf{III. Pappus.}

The references to Pappus in Proclus are not numerous; but we have other evidence that he wrote a commentary on the \emph{Elements}. Thus a scholiast on the definitions of the \emph{Data} uses the phrase ``as Pappus says at the beginning of his (commentary) on the 10th (book) of Euclid\footnote{}.'' Again in the \emph{Fihrist} we are told that Pappus wrote a commentary to the tenth book of Euclid in two parts\footnote{}. Fragments of this still survive in a \textsc{ms}.\ described by Woepcke\footnote{}, Paris.\ No.~952.~2 (supplément arabe de la Bibliothèque impériale), which contains a translation by Abū 'Uthmān (beginning of 10th century) of a Greek commentary on Book \textsc{x}. It is in two books, and there can now be no doubt that the author of the Greek commentary was Pappus\footnote{}. Again Eutocius, in his note on Archimedes, \emph{On the Sphere and Cylinder} \textsc{i}.~13, says that Pappus explained in his commentary on the \emph{Elements} how to inscribe in a circle a polygon similar to a polygon inscribed in another circle; and this would presumably come in his commentary on Book \textsc{xii}., just as the problem is solved in the second scholium on Eucl.\ \textsc{xii}.~1. Thus Pappus' commentary on the \emph{Elements} must have been pretty complete, an additional confirmation of this supposition being forthcoming in the reference of Marinus (a pupil and follower of Proclus) in his preface to the \emph{Data} to ``the commentaries of Pappus on the book\footnote{}.''