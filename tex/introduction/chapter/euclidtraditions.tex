\chapter[Euclid and traditions about him]{Euclid and the traditions about him}

\textsc{As} in the case of the other great mathematicians of Greece, so in Euclid's case, we have only the most meagre particulars of the life and personality of the man.

Most of what we have is contained in the passage of Proclus' summary relating to him, which is as follows\footnote{Proclus, ed.~Friedlein, p.~68, 6---20.}:

``Not much younger than these (sc.~Hermotimus of Colophon and Philippus of Medma) is Euclid, who put together the Elements, collecting many of Eudoxus' theorems, perfecting many of Theaetetus', and also bringing to irrefragable demonstration the things which were only somewhat loosely proved by his predecessors. This man lived\footnote{The word γέγονε must apparently mean ``flourished,'' as Heiberg understands it (\emph{Litterargeschichtliche Studien über Euklid}, 1882, p.~26), not ``was born'', as Hankel took it: otherwise part of Proclus' argument would lose its cogency.} in the time of the first Ptolemy. For Archimedes, who came immediately after the first (Ptolemy)\footnote{So Heiberg understands ἐπιβαλὼν τῷ πρώτῳ (sc.~Πτολεμαίῳ). Friedlein's text has καὶ between ἐπιβαλὼν and τῷ πρώτῳ; and it is right to remark that another reading is καὶ ἐν τῷ πρώτῳ (without ἐπιβαλὼν) which has been translated ``in his first \emph{book},'' by which is understood \emph{On the Sphere and Cylinder}~\textsc{i.}, where (1) in Prop.~2 are the words ``let \emph{BC} be made to \emph{D by the second} (proposition) \emph{of the first} of Euclid's (books),'' and (2) in Prop.~6 the words ``For these things are handed down in the Elements'' (without the name of Euclid). Heiberg thinks the former passage is referred to, and that Proclus must therefore have had before him the words ``by the second of the first of Euclid'': a fair proof that they are genuine, though in themselves they would be somewhat suspicious.}, makes mention of Euclid: and, further, they say that Ptolemy once asked him if there was in geometry any shorter way than that of the elements, and he answered that there was no royal road to geometry\footnote{The same story is told in Stobaeus, \emph{Ecl.}~(\textsc{ii.}~p.~228, 30, ed.~Wachsmuth) about Alexander and Menaechmus. Alexander is represented as having asked Menaechmus to teach him geometry concisely, but he replied: ``O king, through the country there are royal roads and roads for common citizens, but in geometry there is one road for all.''}. He is then younger than the pupils of Plato but older than Eratosthenes and Archimedes; for the latter were contemporary with one another, as Eratosthenes somewhere says.''

This passage shows that even Proclus had no direct knowledge of Euclid's birthplace or of the date of his birth or death. He proceeds by inference. Since Archimedes lived just after the first Ptolemy, and Archimedes mentions Euclid, while there is an anecdote about \emph{some} Ptolemy and Euclid, \emph{therefore} Euclid lived in the time of the first Ptolemy.

We may infer then from Proclus that Euclid was intermediate between the first pupils of Plato and Archimedes. Now Plato died in 347/6, Archimedes lived 287--212, Eratosthenes \emph{c}.~284--204~\textsc{b.c.} Thus Euclid must have flourished \emph{c}.~300~\textsc{b.c.}, which date agrees well with the fact that Ptolemy reigned from 306 to 283~\textsc{b.c.}

It is most probable that Euclid received his mathematical training in Athens from the pupils of Plato; for most of the geometers who could have taught him were of that school, and it was in Athens that the older writers of elements, and the other mathematicians on whose works Euclid's \emph{Elements} depend, had lived and taught. He may himself have been a Platonist, but this does not follow from the statements of Proclus on the subject. Proclus says namely that he was of the school of Plato and in close touch with that philosophy\footnote{Proclus, p.~68, 20, καὶ τῇ προαιρέσει δὲ Πλατωνικός ἐστι καὶ τῇ φιλοσοφίᾳ ταύτῃ οἰκεῖος.}. But this was only an attempt of a New Platonist to connect Euclid with his philosophy, as is clear from the next words in the same sentence, ``for which reason also he set before himself, as the end of the whole Elements, the construction of the so-called Platonic figures.'' It is evident that it was only an idea of Proclus' own to infer that Euclid was a Platonist because his \emph{Elements} end with the investigation of the five regular solids, since a later passage shows him hard put to it to reconcile the view that the construction of the five regular solids was the end and aim of the \emph{Elements} with the obvious fact that they were intended to supply a foundation for the study of geometry in general, ``to make perfect the understanding of the learner in regard to the whole of geometry\footnote{\emph{ibid.}\ p.~71, 8.}.'' To get out of the difficulty he says\footnote{\emph{ibid.}\ p.~70, 19 sqq.} that, if one should ask him what was the aim (σκοπός) of the treatise, he would reply by making a distinction between Euclid's intentions (1)~as regards the subjects with which his investigations are concerned, (2)~as regards the learner, and would say as regards (1)~that ``the whole of the geometer's argument is concerned with the cosmic figures.'' This latter statement is obviously incorrect. It is true that Euclid's \emph{Elements} end with the construction of the five regular solids; but the planimetrical portion has no direct relation to them, and the arithmetical no relation at all; the propositions about them are merely the conclusion of the stereometrical division of the work.

One thing is however certain, namely that Euclid taught, and founded a school, at Alexandria. This is clear from the remark of Pappus about Apollonius\footnote{Pappus, \textsc{vii.}~p.~678, 10---12, συσχολάσας τοῖς ὑπὸ Εὐκλείδου μαθηταῖς ἐν Ἀλεξανδρείᾳ πλεῖστον χρόνον, ὅθεν ἔσχε καὶ τὴν τοιαύτην ἕξιν οὐκ ἀμαθῆ.}: ``he spent a very long time with the pupils of Euclid at Alexandria, and it was thus that he acquired such a scientific habit of thought.''

It is in the same passage that Pappus makes a remark which might, to an unwary reader, seem to throw some light on the personality of Euclid. He is speaking about Apollonius' preface to the first book of his \emph{Conics}, where he says that Euclid had not completely worked out the synthesis of the ``three- and four-line locus,'' which in fact was not possible without some theorems first discovered by himself. Pappus says on this\footnote{Pappus, \textsc{vii.}~pp.~676, 25---678, 6. Hultsch, it is true, brackets the whole passage pp.~676, 25--678, 15, but apparently on the ground of the diction only.}: ``Now Euclid---regarding Aristaeus as deserving credit for the discoveries he had already made in conics, and without anticipating him or wishing to construct anew the same system (such was his scrupulous fairness and his exemplary kindliness towards all who could advance mathematical science to however small an extent), being moreover in no wise contentious and, though exact, yet no braggart like the other [Apollonius]---wrote so much about the locus as was possible by means of the conics of Aristaeus, without claiming completeness for his demonstrations.'' It is however evident, when the passage is examined in its context, that Pappus is not following any tradition in giving this account of Euclid: he was offended by the terms of Apollonius' reference to Euclid, which seemed to him unjust, and he drew a fancy picture of Euclid in order to show Apollonius in a relatively unfavourable light.

Another story is told of Euclid which one would like to believe true. According to Stobaeus\footnote{Stobaeus, \emph{l.c.}}, ``someone who had begun to read geometry with Euclid, when he had learnt the first theorem, asked Euclid, `But what shall I get by learning these things?' Euclid called his slave and said `Give him threepence, since he must make gain out of what he learns.'\thinspace''

In the middle ages most translators and editors spoke of Euclid as Euclid \emph{of Megara}. This description arose out of a confusion between our Euclid and the philosopher Euclid of Megara who lived about 400~\textsc{b.c.} The first trace of this confusion appears in Valerius Maximus (in the time of Tiberius) who says\footnote{\textsc{viii.}~12, ext.~1.} that Plato, on being appealed to for a solution of the problem of doubling the cubical altar, sent the inquirers to ``Euclid the geometer.'' There is no doubt about the reading, although an early commentator on Valerius Maximus wanted to correct ``Eucliden'' into ``\emph{Eudoxum},'' and this correction is clearly right. But, if Valerius Maximus took Euclid the geometer for a contemporary of Plato, it could only be through confusing him with Euclid of Megara. The first specific reference to Euclid as Euclid of Megara belongs to the 14th century, occurring in the ὑπομνηματισμοί of Theodorus Metochita (d.~1332) who speaks of ``Euclid of Megara, the Socratic philosopher, contemporary of Plato,'' as the author of treatises on plane and solid geometry, data, optics etc.: and a Paris \textsc{ms.}\ of the 14th century has ``Euclidis philosophi Socratici liber elementorum.'' The misunderstanding was general in the period from Campanus' translation (Venice 1482) to those of Tartaglia (Venice 1565) and Candalla (Paris 1566). But one Constantinus Lascaris (d.~about 1493) had already made the proper distinction by saying of our Euclid that ``He was different from him of Megara of whom Laertius wrote, and who wrote dialogues''\footnote{Letter to Fernandus Acuna, printed in Maurolycus, \emph{Historia Siciliae}, fol.~21~r.\ (see Heiberg, \emph{Euklid-Studien}, pp.~22---3, 25).}; and to Commandinus belongs the credit of being the first translator\footnote{Preface to translation (Pisauri, 1572).} to put the matter beyond doubt: ``Let us then free a number of people from the error by which they have been induced to believe that our Euclid is the same as the philosopher of Megara'' etc.

Another idea, that Euclid was born at Gela in Sicily, is due to the same confusion, being based on Diogenes Laertius' description\footnote{Diog.~L.~\textsc{ii}.~106, p.~58 ed.~Cobet.} of the philosopher Euclid as being ``of Megara, or, according to some, of Gela, as Alexander says in the Διαδοχαί.''

In view of the poverty of Greek tradition on the subject even as early as the time of Proclus (410--485~\textsc{a.d.}), we must necessarily take \emph{cum grano} the apparently circumstantial accounts of Euclid given by Arabian authors; and indeed the origin of their stories can be explained as the result (1)~of the Arabian tendency to romance, and (2)~of misunderstandings.

We read\footnote{Casiri, \emph{Bibliotheca Arabico-Hispana Escurialensis}, \textsc{i}.~p.~339. Casiri's source is al-Qifṭī (d.~1248), the author of the \emph{Ta'rīkh al-Ḥukamā}, a collection of biographies of philosophers, mathematicians, astronomers etc.} that ``Euclid, son of Naucrates, grandson of Zenarchus\footnote{The \emph{Fihrist} says ``son of Naucrates, the son of Berenice (?)'' (see Suter's translation in \emph{Abhandlungen sur Gesch.~d.~Math.}~\textsc{vi}.~Heft, 1892, p.~16).}, called the author of geometry, a philosopher of somewhat ancient date, a Greek by nationality domiciled at Damascus, born at Tyre, most learned in the science of geometry, published a most excellent and most useful work entitled the foundation or elements of geometry, a subject in which no general treatise existed before among the Greeks: nay, there was no one even of later date who did not walk in his footsteps and frankly profess his doctrine. Hence also Greek, Roman and Arabian geometers not a few, who undertook the task of illustrating this work, published commentaries, scholia, and notes upon it, and make an abridgment of the work itself. For this reason the Greek philosophers used to post up on the doors of their schools the well-known notice: `Let no one come to our school, who has not first learned the elements of Euclid.'\thinspace'' The details at the beginning of this extract cannot be derived from Greek sources, for even Proclus did not know anything about Euclid's father, while it was not the Greek habit to record the name of grandfathers, as the Arabians commonly did. Damascus and Tyre were no doubt brought in to gratify a desire in which the Arabians always showed to connect famous Greeks in some way or other with the East. Thus Naṣīraddīn, the translator of the \emph{Elements}, who was of Ṭūs in Khurāsān, actually makes Euclid out to have been ``Thusinus'' also\footnote{The same predilection made the Arabs describe Pythagoras as a pupil of the wise Salomo, Hipparchus as the exponent of Chaldaean philosophy of as the Chaldaean, Archimedes as an Egyptian etc.\ (Ḥājī Khalfa, \emph{Lexicon Bibliographicum}, and Casiri).}. The readiness of the Arabians to run away with an idea is illustrated by the last words of the extract. Everyone knows the story of Plato's inscription over the porch of the Academy: ``let no one unversed in geometry enter my doors''; the Arab turned geometry into \emph{Euclid's} geometry, and told the story of Greek philosophers in general and ``\emph{their} Academies.''

Equally remarkable are the Arabian accounts of the relation of Euclid and Apollonius\footnote{The authorities for these statements quoted by Casiri and Ḥājī Khalfa are al-Kindī's tract \emph{de instituto libri Euclidis} (al-Kindī died about 873) and a commentary by Qāḍīzāde ar-Rūmī (d.~about 1440) on a book called \emph{Ashkāl at-ta' sīs} (fundamental propositions) by Ashraf Shamsaddīn as-Samarqandī (\emph{c}.~1276) consisting of elucidations of 35 propositions selected from the first books of Euclid. Naṣīraddīn likewise says that Euclid cut out two of 15 books of elements then existing and published the rest under his own name. According to Qāḍīzāde the king heard that there was a celebrated geometer named Euclid at \emph{Tyre}: Naṣīraddīn says that he sent for Euclid of Ṭūs.}. According to them the \emph{Elements} were originally written, not by Euclid, but by a man whose name was Apollonius, a carpenter, who wrote the work in 15 books or sections\footnote{So says the \emph{Fihrist}. Suter (\emph{op.~cit.}~p.~49) thinks that the author of the \emph{Fihrist} did not suppose Apollonius \emph{of Perga} to be the writer of the \emph{Elements}, as later Arabian authorities did, but that he distinguished another Apollonius whom he calls ``a carpenter.'' Suter's argument is based on the fact that the \emph{Fihrist's} article on Apollonius (of Perga) says nothing of the \emph{Elements}, and that it gives the three great mathematicians, Euclid, Archimedes and Apollonius, in the correct chronological order.}. In the course of time some of the work was lost and the rest became disarranged, so that one of the kings at Alexandria who desired to study geometry and to master this treatise in particular first questioned about it certain learned men who visited him and then sent for Euclid who was at that time famous as a geometer, and asked him to revise and complete the work and reduce it to order. Euclid then re-wrote it in 13 books which were thereafter known by his name. (According to another version Euclid composed the 13 books out of commentaries which he had published on two books of Apollonius on conics and out of introductory matter added to the doctrine of the five regular solids.) To the thirteen books were added two more books, the work of others (though some attribute these also to Euclid) which contain several things not mentioned by Apollonius. According to another version Hypsicles, a pupil of Euclid at Alexandria, offered to the king and published Books~\textsc{xiv.}\ and \textsc{xv.}, it being also stated that Hypsicles had ``discovered'' the books, by which it appears to be suggested that Hypsicles had edited them from materials left by Euclid.

We observe here the correct statement that Books~\textsc{xiv.}\ and \textsc{xv.}\ were not written by Euclid, but along with it the incorrect information that Hypsicles, the author of Book~\textsc{xiv.}, wrote Book~\textsc{xv.}\ also.

The whole of the fable about Apollonius having preceded Euclid and having written the \emph{Elements} appears to have been evolved out of the preface to Book~\textsc{xiv.}\ by Hypsicles, and in this way; the Book must in early times have been attributed to Euclid, and the inference based upon the assumption was left uncorrected afterwards when it was recognised that Hypsicles was the author. The preface is worth quoting:

``Basilides of Tyre, O Protrarchus, when he came to Alexandria and met my father, spent the greater part of his sojourn with him on account of their common interest in mathematics. And once, when examining the treatise written by Apollonius about the comparison between the dodecahedron and the icosahedron inscribed in the same sphere, (showing) what ratio they have to one another, they thought that Apollonius had not expounded this matter properly, and accordingly they emended the exposition, as I was able to learn from my father. And I myself, later, fell in with another book published by Apollonius, containing a demonstration relating to the subject, and I was greatly interested in the investigation of the problem. The book published by Apollonius is accessible to all---for it has a large circulation, having apparently been carefully written out later---but I decided to send you the comments which seem to me to be necessary, for you will through your proficiency in mathematics in general and in geometry in particular form an expert judgment on what I am about to say, and you will lend a kindly ear to my disquisition for the sake of your friendship to my father and your goodwill to me.''

The idea that Apollonius preceded Euclid must evidently have been derived from the passage just quoted. It explains other things besides. Basilides must have been confused with βασιλεύς, and we have a probable explanation of the ``Alexandrian king,'' and of the ``learned men who visited'' Alexandria. It is possible also that in the ``Tyrian'' of Hypsicles' preface we have the origin of the notion that Euclid was born in Tyre. These inferences argue, no doubt, very defective knowledge of Greek: but we could expect no better from those who took the \emph{Organon} of Aristotle to be ``instrumentum musicum pneumaticum,'' and who explained the name of Euclid, which they variously pronouned as \emph{Uclides} or \emph{Icludes}, to be compounded of \emph{Ucli} a key, and \emph{Dis} a measure, or, as some say, geometry, so that \emph{Uclides} is equivalent to the \emph{key of geometry}!

Lastly, the alternative version, given in brackets above, which says that Euclid made the \emph{Elements} out of commentaries which he wrote on two books of Apollonius on conics and prolegomena added to the doctrine of the five solids, seems to have arisen, through a like confusion, out of a later passage\footnote{Heiberg's Euclid, vol.~\textsc{v}.~p.~6.} in Hypsicles' Book~\textsc{xiv.}: ``And this is expounded by Aristaeus in the book entitled `Comparison of the five figures,' and by Apollonius in the second edition of his comparison of the dodecahedron with the icosahedron.'' The ``doctrine of the five solids'' in the Arabic must be the ``Comparison of the five figures'' in the passage of Hypsicles, for nowhere else have we any information about a work bearing this title, nor can the Arabians have had. The reference to the \emph{two books} of Apollonius on \emph{conics} will then be the result of mixing up the fact that Apollonius wrote a book on conics with the \emph{second edition} of the other work mentioned by Hypsicles. We do not find elsewhere in Arabian authors any mention of a commentary by Euclid on Apollonius and aristaeus: so that the story in the passage quoted is really no more than a variation of the fable that the \emph{Elements} were the work of Apollonius.