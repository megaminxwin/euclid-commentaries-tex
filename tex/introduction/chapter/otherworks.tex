\chapter{Euclid's other works}

\textsc{In} giving a list of the Euclidean treatises other than the \emph{Elements}, I shall be brief: for fuller accounts of them, or speculations with regard to them, reference should be made to the standard histories of mathematics\footnote{}.

I will take first the works which are mentioned by Greek authors.

1. The \emph{Pseudaria}.

I mention this first because Proclus refers to it in the general remarks in praise of the \emph{Elements} which he gives immediately after the mention of Euclid in his summary. He says\footnote{}: ``But, inasmuch as many things, while appearing to rest on truth and to follow from scientific principles, really tend to lead one astray from the principles and deceive the more superficial minds, he has handed down methods for the discriminative understanding of these things as well, by the use of which methods we shall be able to give beginners in this study practice in the discovery of paralogisms, and to avoid being misled. This treatise, by which he puts this machinery in our hands, he entitled (the book) of Pseudaria, enumerating in order their various kinds, exercising our intelligence in each case by theorems of all sorts, setting the true side by side with the false, and combining the refutation of error with practical illustration. This book then is by way of cathartic and exercise, while the Elements contain the irrefragable and complete guide to the actual scientific investigation of the subjects of geometry.''

The book is considered to be irreparably lost. We may conclude however from the connexion of it with the \emph{Elements} and the reference to its usefulness for beginners that it did not go outside the domain of elementary geometry\footnote{}.

2. The \emph{Data}.

The \emph{Data} (δεδομένα) are included by Pappus in the \emph{Treasury of Analysis} (τόπος ἀναλυόμενος), and he describes their contents\footnote{}. They are still concerned with elementary geometry, though forming part of the introduction to higher analysis. Their form is that of propositions proving that, if certain things in a figure are given (in magnitude, in species, etc.), something else is given. The subject-matter is much the same as that of the planimetrical books of the \emph{Elements}, to which the \emph{Data} are often supplementary. We shall see this later when we come to compare the propositions in the \emph{Elements} which give us the means of solving the general quadratic equation with the corresponding propositions of the \emph{Data} which give the solution. The \emph{Data} may in fact be regarded as elementary exercises in analysis.

It is not necessary to go more closely into the contents, as we have the full Greek text and the commentary by Marinus newly edited by Menge and therefore easily accessible\footnote{}.

3. The book \emph{On divisions (of figures)}.

This work (περὶ διαιρέσεων βιβλίον) is mentioned by Proclus\footnote{}. In one place he is speaking of the conception or definition (λόγος) of \emph{figure}, and of the divisibility of a figure into others differing from it in kind; and he adds: ``For the circle is divisible into parts unlike in definition or notion (ἀνόμοια τῷ λόγῳ), and so is each of the rectilineal figures; this is in fact the business of the writer of the Elements in his Divisions, where he divides given figures, in one case into like figures, and in another into unlike\footnote{}.'' ``Like'' and ``unlike'' here mean, not ``similar'' and ``dissimilar'' in the technical sense, but ``like'' or ``unlike \emph{in definition} or \emph{notion}'' (λόγῳ): thus to divide a triangle into triangles would be to divide it into ``like'' figures, to divide a triangle into a triangle and a quadrilateral would be to divide it into ``unlike'' figures.

The treatise is lost in Greek but has been discovered in the Arabic. First John Dee discovered a treatise \emph{De divisionibus} by one Muḥammad Bagdadinus\footnote{} and handed over a copy of it (in Latin) in 1563 to Commandinus, who published it, in Dee's name and his own, in 1570\footnote{}. Dee did not himself translate the tract from the Arabic; he found it in Latin in a \textsc{ms.}\ which was then in his own possession but was about 20 years afterwards stolen or destroyed in an attack by a mob on his house at Mortlake\footnote{}. Dee, in his preface addressed to Commandinus, says nothing of his having \emph{translated} the book, but only remarks that the very illegible \textsc{ms.}\ had caused him much trouble and (in a later passage) speaks of ``the actual, very ancient, copy from which I \emph{wrote out}...'' (in ipso unde descripsi vetustissimo exemplari). The Latin translation of this tract from the Arabic was probably made by Gherard of Cremona (1114--1187), among the list of whose numerous translations a ``liber divisionum'' occurs. The Arabic original cannot have been a direct translation from Euclid, and probably was not even a direct adaptation of it; it contains mistakes and unmathematical expressions, and moreover does not contain the propositions about the division of a circle alluded to by Proclus. Hence it can scarcely have contained more than a fragment of Euclid's work.

But Woepcke found in a \textsc{ms.}\ at Paris a treatise in Arabic on the division of figures, which he translated and published in 1851\footnote{}. It is expressly attributed to Euclid in the \textsc{ms.}\ and corresponds to the description of it by Proclus. Generally speaking, the divisions are divisions into figures of the same kind as the original figures, e.g.~of triangles into triangles; but there are also divisions into ``unlike'' figures, e.g.~that of a triangle by a straight line parallel to the base. The missing propositions about the division of a circle are also here: ``to divide into two equal parts a given figure bounded by an arc of a circle and two straight lines including a given angle'' and ``to draw in a given circle two parallel straight lines cutting off a certain part of the circle.'' Unfortunately the proofs are given of only four propositions (including the two last mentioned) out of 36, because the Arabic translator found them too easy and omitted them. To illustrate the character of the problems dealt with I need only take one more example: ``To cut off a certain fraction from a (parallel-) trapezium by a straight line which passes through a given point lying inside or outside the trapezium but so that a straight line can be drawn through it cutting both the parallel sides of the trapezium.'' The genuineness of the treatise edited by Woepcke is attested by the facts that the four proofs which remain are elegant and depend on propositions in the \emph{Elements}, and that there is a lemma with a true Greek ring: ``to apply to a straight line a rectangle equal to the rectangle contained by \emph{AB, AC and deficient by a square.}'' Moreover the treatise is no fragment, but finishes with the words ``end of the treatise,'' and is a well-ordered and compact whole. Hence we may safely conclude that Woepcke's is not only Euclid's own work but the whole of it. A restoration of the work, with proofs, was attempted by Ofterdinger\footnote{}, who however does not give Woepcke's props.~30, 31, 34, 35, 36. We have now a satisfactory restoration with ample notes and an introduction, by R.~C.~Archibald, who used for the purpose Woepcke's text and a section of Leonardo of Pisa's \emph{Practica geometriae} (1220)\footnote{}.

4. The \emph{Porisms}.

It is not possible to give in this place any account of the controversies about the contents and significance of the three lost books of Porisms, or of the important attempts by Robert Simson and Chasles to restore the work. These may be said to form a whole literature, references to which will be found most abundantly given by Heiberg and Loria, the former of whom has treated the subject from the philological point of view, most exhaustively, while the latter, founding himself generally on Heiberg, has added useful details, from the mathematical side, relating to the attempted restorations, etc.\footnote{} It must suffice here to give an extract from the only original source of information about the nature and contents of the \emph{Porisms}, namely Pappus\footnote{}. In his general preface about the books composing the \emph{Treasury of Analysis} (τόπος ἀναλυόμενος) he says:

``After the Tangencies (of Apollonius) come, in three books, the Porisms of Euclid, [in the view of many] a collection most ingeniously devised for the analysis of the more weighty problems, [and] although nature presents an unlimited number of such porisms\footnote{}, [they have added nothing to what was written originally by Euclid, except that some before my time have shown their want of taste by adding to a few (of the propositions) second proofs, each (proposition) admitting of a definite number of demonstrations, as we have shown, and Euclid having given one for each, namely that which is the most lucid. These porisms embody a theory subtle, natural, necessary, and of considerable generality, which is fascinating to those who can see and produce results].

``Now all the varieties of porisms belong, neither to theorems nor problems, but to a species occupying a sort of intermediate position [so that their enunciations can be formed like those of either theorems or problems], the result being that, of the great number of geometers, some regarded them as of the class of theorems, and others of problems, looking only to the form of the proposition. But that the ancients knew better the difference between these three things is clear from the definitions. For they said that a theorem is that which is proposed with a view to the demonstration of the very thing proposed, a problem that which is thrown out with a view to the construction of the very thing proposed, and a porism that which is proposed with a view to the producing of the very thing proposed. [But this definition of the porism was changed by the more recent writers who could not produce everything, but used these elements and proved only the fact that that which is sought really exists, but did not produce it\footnote{} and were accordingly confuted by the definition and the whole doctrine. They based their definition on an incidental characteristic, thus: A porism is that which falls short of a locus-theorem in respect of its hypothesis\footnote{}. Of this kind of porisms loci are a species, and they abound in the Treasury of Analysis; but this species has been collected, named and handed down separately from the porisms, because it is more widely diffused than the other species]. But it has further become characteristic of porisms that, owing to their complication, the enunciations are put in a contracted form, much being by usage left to be understood; so that many geometers understand them only in a partial way and are ignorant of the more essential features of their contents.

``[Now to comprehend a number of propositions in one enunciation is by no means easy in these porisms, because Euclid himself has not in fact given many of each species, but chosen, for examples, one or a few out of a great multitude\footnote{}. But at the beginning of the first book he has given some propositions, to the number of ten, of one species, namely that more fruitful species consisting of loci.] Consequently, finding that these admitted of being comprehended in one enunciation, we have set it out thus: